\documentclass[UTF8]{ctexart}
%%%%%%%%%%%%%%%%%%%%%%%%%%%== 引入宏 ==%%%%%%%%%%%%%%%%%%%%%%%%%%%%%
\usepackage{cite}
\usepackage{amsmath}	% 使用数学公式
\usepackage{graphicx}	% 插入图片/PDF/EPS 等图像
\usepackage{subfigure}	% 使用子图像或者子表格
\usepackage{geometry}	% 设置页边距
\usepackage{fancyhdr}	% 设置页眉页脚
\usepackage{setspace}	% 设置行间距
\usepackage{hyperref}	% 让生成的文章目录有链接,点击时会自动跳转到该章节
\usepackage{url}
\usepackage{caption2}

%%%%%%%%%%%%%%%%%%%%%%%%%%== 设置全局环境 ==%%%%%%%%%%%%%%%%%%%%%%%%%%%%
% [geometry] 设置页边距
\geometry{top=2.6cm, bottom=2.6cm, left=2.45cm, right=2.45cm, headsep=0.4cm, foot=1.12cm}
% 设置行间距为 1.5 倍行距
\onehalfspacing
% 设置页眉页脚
\pagestyle{fancy}
%\lhead{左头标}
%\chead{\today}
%\rhead{152xxxxxxxx}
\lfoot{}
\cfoot{\thepage}
\rfoot{}
%\renewcommand{\headrulewidth}{0.4pt}
%\renewcommand{\headwidth}{\textwidth}
%\renewcommand{\footrulewidth}{0pt}

%%%%%%%%%%%%%%%%%%%%%%%%%%== 自定义命令  ==%%%%%%%%%%%%%%%%%%%%%%%%%%%%%%
% 此行使文献引用以上标形式显示
\newcommand{\supercite}[1]{\textsuperscript{\cite{#1}}}
% 此行使section中的图、表、公式编号以A-B的形式显示
\renewcommand{\thetable}{\arabic{section}-\arabic{table}}
\renewcommand{\thefigure}{\arabic{section}-\arabic{figure}}
\renewcommand{\theequation}{\arabic{section}-\arabic{equation}}
% 此行使图注、表注与编号之间的分隔符缺省,默认是冒号:
\renewcommand{\captionlabeldelim}{~}

%===================================== 标题设置  ==========================================
% \heiti \kaishu 为字体设置,ctex 会自动根据操作系统加载字体
\title{\huge{\heiti 基于MPI的相似轨迹计算}}
\author{\small{\kaishu 吉吉}\\[2pt]
\small{\kaishu 加里顿大学计算机科学与技术学院}\\[2pt]
\small{Email:}
\url{csyiji@gmail.com}
}
\date{} % 去除默认日期
%\date{\today}

%===================================== 正文区域  ==========================================
\begin{document}
\maketitle
% \tableofcontents % 目录内容,注释取消掉可实现目录


\section{重点难点分析}\label{sec2}
\subsection{文件数据的解析}

我们实验的数据由10000个轨迹数据文件和1个索引文件。其整个执行流程如下:

\begin{itemize}
	\item 假设目标轨迹为A,则先根据A的轨迹编号在索引文件中查询,得到行号x与列号y;
	\item 轨迹A数据所在文件名为x\_data ,读取此文件第y行即为轨迹A的内容;
	\item 轨迹的内容结构为:轨迹编号+轨迹数量+轨迹序列..,读取轨迹序列存入二维数组,再进行后续处理。
\end{itemize}

\subsection{LCSS算法阈值选择}

由LCSS算法\ref{sec1:subsec3:eq2}的定义说明,其中计算的判断条件在于轨迹点距离$\epsilon$和轨迹点顺序差$\delta$两个阈值的选择。
而此两个阈值不能仅仅设为常数,它们会随着目标轨迹的变化而变化。因此:

\begin{itemize}
	\item $\epsilon$:选取轨迹上每两个相邻点间的欧式距离和的$\frac{1}{k}$
	\item $\delta$:选取轨迹上点数目的$\frac{1}{k}$
\end{itemize}

\subsection{相似度算法的非递归方式实现}
\label{sec2:subsec3}

\begin{figure}[!htbp]
  \centering
  \includegraphics[width=0.35\textwidth]{fig/fig01.pdf}\\
  \caption{LCSS非递归实现}
  \label{sec2:subsec3:fg1}
\end{figure}

LCSS公式\ref{sec1:subsec3:eq1} 给出的是求临近点的递归实现的方式,而我们的每条轨迹点约有60~100个点,则任意两条轨迹计算的递归深度至少为3600,占用的栈空间很大,
因此我们采用非递归方式实现求临近点个数的算法。图\ref{sec2:subsec3:fg1}展示了非递归的实现方式:

\begin{itemize}
	\item 用矩阵来存储任意两点的计算结果,计算主要遵循两个原则:
	\item 轨迹上两点若满足公式里的阈值范围,则将矩阵左上角数值加1;
	\item 轨迹上两点若不满足阈值范围,则选择当前矩阵点的上方和左方的最大的数值作为当前值。
\end{itemize}
\end{document}